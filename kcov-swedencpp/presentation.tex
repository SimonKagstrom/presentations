\documentclass{beamer}


\usepackage[utf8]{inputenc}
\usepackage{pgfpages}
\usepackage{xcolor}

\setbeameroption{show notes}
\setbeameroption{show notes on second screen=left}

\usetheme{Warsaw}
\graphicspath{ {./images/}{../images/} }

\AtBeginSection[specialframe]
{
  \begin{frame}{Table of Contents}
   \tableofcontents[currentsection]
  \end{frame}
}

\title[Kcov - a single-step code coverage tool] %optional
{Kcov - a single-step code coverage tool}

%\subtitle{A short story}

\author{Simon Kågström}

\institute
{
  Net Insight\\
  \texttt{https://github.com/SimonKagstrom/kcov}
}

%\logo{\includegraphics[height=1.5cm]{lion-logo.png}}

\begin{document}

\begin{frame}
  \titlepage
  \note{My name is Simon Kågström and I work at Net Insight as a developer of embedded software. Tonight I will present kcov, a code coverage collection and presentation tool for UNIX systems.

So let's get started!}
\end{frame}

\begin{frame}
  \frametitle{Motivation}
  \note<1->{
    \footnotesize
    I'll start with some motivation for the talk. Code coverage, as described by wikipedia, is the degree to which the source code of a program is executed when a particular test suite executes, measured in percentage. Code coverage tools can typically show which source code lines which have been executed, and sometimes how many times they are executed.

    So let's say you're the developer of a particular piece of code, as shown on the slide. You run some tests on it, and collect code coverage to see how well your test suite actually works. NEXT.
  }

  \note<2>{
    After code coverage collection, you will get a report which looks something like this. Red lines denote executable lines which have not been executed, green lines are executed lines. White lines are non-executable lines.

    Hmm... Hang on now, why is everything between line 128 and 143 marked as non-executable? Some of you might already have identified the code in question. This is the SSL/TLS bug known as ``goto fail''. Looking at line 127, you can see that the control flow will unconditionally jump to the fail label, so lines 128 to 143 are indeed unreachable.

    Stands out pretty clear from the coverage report don't you think? Modern compilers will also issue a warning on code like this.}

  \includegraphics<1>[height=8cm]{goto_fail_no_coverage}
  \includegraphics<2>[height=8cm]{goto_fail}
\end{frame}

\begin{frame}
  \frametitle{More motivation}

  \note{We programmers should be humble people. And why is that?
    Because we all fail at times, and shoot ourselves in our
    feet. I've done this myself many times. I've failed with raw
    pointers, with shared pointers, from lambdas, using the Linker and
    so on. I've even shot myself in the foot using Java bytecode! So I
    try to be humble about my abilities, I know I will fail again and again.

    However, this is why I believe we need good tooling to assist
    us. Code coverage is one such tool, together with for example the
    clang sanitizers, good debuggers and so on.

    It's also the reason why good methodology is important, through
    for example test driven development and unit testing. Especially
    when developing unit tests, I believe code coverage collection has
    a lot to offer, and it's the area where I personally use it most. Pretty much daily in fact.
  }
  \begin{itemize}
    \item As per the last example, the main motivation for code coverage is to detect unexecuted code
  \end{itemize}
\end{frame}

\begin{frame}{Outline}
  \tableofcontents
  \note{This is the outline of the rest of the talk. I will start with a short discussion of problems and limitations with traditional tools, then introduce kcov and demonstrate some of the main features.

    I will then go into a bit of details about how kcov has been implemented, and give some lessons learned in the process.

  Finally, I will discuss some other features of kcov, which are perhaps a bit less relevant on a C++ workshop.}
\end{frame}

\section{Overview}
\subsection{Problems with traditional tools}
\begin{frame}[fragile]{Problems with traditional tools}
  \note{On UNIX, the traditional way of collecting code coverage has been to use gcov for collection, and lcov for HTML presentation, if desired. The process is a bit cumbersome, as can be seen in the example below. You first need a special --coverage option, which produces extra metadata files in the build directory. After running the program, you then get data for the actual execution in the build directory.

    After this, you need to run the presentation tool, like lcov, to get actual output. If the process crashes during execution, you will lose the coverage output altogether. Gcov also doesn't gather code coverage from shared libraries.

    Fortunately, using kcov does away with all these disadvantages, allowing collection without special compiler options, reporting without droppings and all done in a single step. Quite a sales pitch, don't you agree? We should note here that there are other collection methods, for example via clang and obviously a set of commercial tools.}
  \begin{itemize}
  \item gcov + lcov is a multi-step process
  \item gcov leaves droppings after compilation/running
  \item A program which crashes will not generate coverage data
  \end{itemize}
  \begin{Example}
    \begin{semiverbatim}
     \scriptsize
\$ gcc -g -Wall --coverage goto-fail.c
\$ ./a.out
\$ ls
  a.out  goto-fail.c  goto-fail.gcda  goto-fail.gcno
\$ lcov --capture --directory project-dir --output-file coverage.info
\$ genhtml coverage.info --output-directory out
    \end{semiverbatim}
  \end{Example}
\end{frame}

\subsection{Kcov overview}
\begin{frame}{Kcov overview}
  \begin{itemize}
  \item Kcov started as a fork of Bcov by Thomas Neumann in 2010
  \item Bcov doesn't rely on compile-time instrumentation, but instead uses DWARF debug data
  \item Bcov was a great idea, but I thought it could be improved upon
  \item It is not related to the kernel Kcov (and predates it by many years)
  \item<2-> Can you guess why it's called kcov?
  \end{itemize}

  \note{
    Kcov started as a fork of bcov, which doesn't rely on compile-time instrumentation of binaries, but instead uses the DWARF debug information present as long as you compile with -g. It can also produce lcov-like output.

    I thought bcov was a great idea, but it was still somewhat cumbersome to use, separating collection and reporting. The codebase was a bit difficult for me to follow, so I forked it instead of improving on the original project. Today I would problably not have done it that way.

    Guess the name! The original idea I had was to use kernel kprobes, which allows setting breakpoints on kernel code, while still retaining the userspace functionality. K therfore stands for kernel.

    The kernel functionality never worked very well though.
  }
\end{frame}

\subsection{Main features of kcov}
\begin{frame}{Interactive Demo!}
  \note{
      \begin{itemize}
      \item Demo: Make sure /tmp/kcov is empty

      \item kcov /tmp/kcov projects/build/kcov/target/src/kcov

      \item Show result in browser

      \item Show how a single file looks, describe 1/3, execution order etc

      \item Show file list again, note /usr/include etc

      \item Remove /tmp/kcov. All output is placed in the out-directory, so this cleans up everything from the coverage run

      \item Run again with --include-pattern

      \item Run calc, show merged report

      \item Run kcov with --include-pattern, note that --exclude-pattern and paths also exist

      \item Run kcov on another program (calc?), show that output is merged
    \end{itemize}
  }
\end{frame}

\subsection{Integration with CI systems}
\begin{frame}{Integration with CI systems}
  \note{Kcov also integrates with several systems and sites for continous integration. Jenkins has a plugin for Cobertura, normally a Java code coverage collection tool. As kcov produces XML output for Cobertura, it's easy to integrate in that environment. SonarQube is handled in a similar way.

 It also supports some popular cloud services. NEXT. It can upload directly to coveralls, which is often used together with travis-ci. Coveralls is an easy way to get fancy web stats for your project coverage. NEXT. Similarly, codecov.io is also easy to integrate, but supports kcov directly from the upstream project.

Personally, I haven't picked a favourite cloud coverage service just yet, but instead use both with travis-ci and added dual badges on github.
  }
  \begin{itemize}
  \item Jenkins and SonarQube output is generated by kcov
  \item \textcolor<2>{red}{Uploading to Coveralls.io is built-in}
  \item \textcolor<3>{red}{Uploading to Codecov.io is supported by the upstream project}
  \end{itemize}
  \includegraphics<1-2>[height=6cm]{coveralls}
  \includegraphics<3>[height=6cm]{codecov}
\end{frame}

\subsection{Python/Bash}
\begin{frame}{Python and Bash code coverage}
  \begin{itemize}
    \item Kcov can also collect coverage for Python
  \end{itemize}

  \includegraphics[width=7cm]{bash}\includegraphics[width=5cm]{python}
\end{frame}

\section{How kcov works}

\subsection{Elves, dwarves and breakpoints}
\begin{frame}{Dwarf stabs Mach-O Elf}
  \note{\footnotesize Lets start with an overview of technologies used by
    kcov. Binaries in Linux are nowadays almost always stored in the
    ELF binary format. ELF binaries contain sections which contain
    code, data, constants, relocation information and so on. Where
    relevant, the sections contain a memory address where the code or
    data is loaded into memory during execution.

    On Mac OS X, Mach-O is used as the binary format, and functions in
    pretty much the same way as ELF. Windows has some other format.

    The most important of these for kcov is Dwarf, however. Dwarf is a
    format for debug information, and is related but independent from
    the binary formats. Dwarf sections are present in both ELF
    binaries, and Mach-O dittos and contain information needed by
    debuggers. For example, the mapping between source lines and
    addresses is needed when you set a breakpoint on a source line,
    variable type information when you print variables and so on.

    I guess you've noted by now that the world of binary file formats
    and debug information is full of funny names. Stabs is an older
    format for debug information, seldomly used today.

  We'll continue with how kcov uses these on the next slide.}

  \begin{itemize}
  \item \textbf{ELF} is a binary format used on many Unices
    \begin{itemize}
    \item Both for object files, shared libraries and executable files
    \item Contains the code, data, constants and relocation information
    \end{itemize}

  \item \textbf{Mach-O} is the binary format used on Mac OS X
  \item \textbf{Dwarf} is a format for debug information
    \begin{itemize}
    \item Contains the mapping between source lines and addresses
    \item Type information for variables etc
    \end{itemize}
  \end{itemize}

\end{frame}

\begin{frame}[fragile]{Instrumenting binaries with kcov}
  \note{
    Kcov relies on Dwarf debug information to set a breakpoint on each executable line in the program. The debug information contains records of file:line to address mappings, which is exactly what kcov needs. NEXT. In the disassembly output, you can see this as the instructions marked in red, which correspond to DWARF entries.
% Disassembly here
    
    Kcov executes the program in stopped state via ptrace, sets all breakpoints and then lets the program continue again. On each executed breakpoint, kcov marks the file:line pair as executed, removes the breakpoint and lets the program continue again. That's the basic way kcov works.
  }
  \begin{Example}
    \begin{semiverbatim}
      \scriptsize
 124:     return (void*)op\_create(RPAR);
400ff7:       \textcolor<2>{red}{bf 30 00 00 00}          mov    \$0x30,\%edi
400ffc:       e8 1f 05 00 00          callq  401520 <op\_create>
401001:       e9 20 ff ff ff          jmpq   400f26 <ts\_next\_token+0x46>
401006:       66 2e 0f 1f 84 00 00    nopw   \%cs:0x0(\%rax,\%rax,1)
40100d:       00 00 00
 125: p\_state->last\_token\_is\_od = 0;
401010:       \textcolor<2>{red}{c7 43 08 00 00 00 00}    movl   \$0x0,0x8(\%rbx)
 126: return (void*)op\_create(op);
401017:       \textcolor<2>{red}{bf 04 00 00 00}          mov    \$0x4,\%edi
40101c:       e8 ff 04 00 00          callq  401520 <op\_create>
   \end{semiverbatim}
   \end{Example}
 % Disassembly
  % file:line -> addr
  % Cleared once it has been executed
\end{frame}

\subsection{Implementation quirks on different architectures}
\begin{frame}[fragile]{kcov on Linux, FreeBSD and Mac OSX}
  \note{There is no fully standardized way of setting and controlling breakpoints on UNIX. The ancient ptrace is typically used for this, through PEEKTEXT/POKETEXT options. The original bcov implementation is the base for the Linux implementation, but Alan Somers have more recently extracted the OS-dependent parts and ported Kcov to FreeBSD. On Linux and FreeBSD, libelf and elfutils is used for parsing binaries.

OSX works in an entirently different way though. The key difference between OSX and Linux/FreeBSD is that OSX uses it's own binary format, Mach-O. So we now have Dwarves, Elves and Mach-O. For ELF, we have libelf to do the parsing, but Apple doesn't publish a library for Mach-O. I therefore opted on a simpler solution on OSX. The LLDB debugger comes with the development environment on OSX, and it conveniently offers a nice C++ API.

So on OSX, the entire parsing, process control and breakpoint setting is only 486 lines of code. The disadvantage with the LLDB implementation is that it's significantly slower than using regular ptrace. During normal debugging, you typically don't set tens of thousands of breakpoints!
  }
    \begin{itemize}
    \item ptrace is a truly archaic interface, and pretty non-portable
    \item libelf is also interesting: \textbf{elf\_version} must be called at program start!
    \item Mac OSX has been implemented using the LLDB debugger as a library
    \end{itemize}
  \begin{Example}
    \begin{semiverbatim}
      \scriptsize
static long getRegs(pid\_t pid, void *addr, void *regs, size\_t len)
\{
#if defined(\_\_aarch64\_\_)
    struct iovec iov = \{regs, len\};
    return ptrace(PTRACE\_GETREGSET, pid, (void *)NT\_PRSTATUS, &iov);
#else
    return ptrace((\__ptrace\_request ) PTRACE\_GETREGS, pid, NULL, regs);
#endif
\}
    \end{semiverbatim}
  \end{Example}
\end{frame}

\begin{frame}{Design}
  \note{
    \footnotesize
    Kcov is written in C++, but unfortunately it's C++03, so a bit unmodern for these kinds of meetings! I'll just shortly discuss the kcov design a bit.

The design revolves around a set of abstract interfaces, as show on the figure and through heavy use of the observer pattern. Binary files are parsed via the IFileParser interface. Binary execution is handled through the IEngine interface, and output is handled via the IWriter interface.

Control is handled through the collector, which receives file/line to address mappings from the file parser. The collector will then set breakpoints, execute the program, and collect breakpoint hits through the engine. The reporter receives and memorizes executed lines and reports them to the writers, which produces the output. Pretty simple, don't you think?

The design was originally made to accomodate ELF and ptrace, and the only true File parser is the Elf parser. When kcov is used with Python, bash or indeed on OSX via LLDB, parsing and execution is kept together, so these implement both the engine and the file parser interfaces.}
  \includegraphics[width=\linewidth]{design}
\end{frame}

\subsection{ELF/Dwarf implementation quirks}
\begin{frame}[fragile]{Dwarf quirks}
  \note{
    \footnotesize
    There are some interesting quirks and bugs concerning Dwarf usage in Linux. The most irritating and interesting one is when the file:line to address entries contain invalid entries. We again have the disassembly output you saw before. NEXT. However, if you look at the second entry marked in red here, you see that it points into the middle of an instruction.

    This is possible on x86 since it has variable instruction length, so an instruction can start on any byte address. Now, a breakpoint on x86 is simply a special instruction, encoded with a single byte, 0xcc hex. Setting a breakpoint therefore means overwriting the start of the instruction you want to break at with 0xcc. But if 0xcc is written into the middle of an instruction, you either get an invalid instruction, or even worse, a valid but unintended instruction.

    kcov works around this by simply disassembing the entire program, and discarding Dwarf entries which doesn't point to the start of an instruction. This is somewhat expensive, so it's only enabled with an option. On architectures which have a fixed instruction length, it instead discards entries which are unaligned.}
  \begin{itemize}
    \item Dwarf generation on Linux is sometimes buggy, containing invalid entries
  \end{itemize}
  \begin{Example}
    \begin{semiverbatim}
      \scriptsize
 124:     return (void*)op\_create(RPAR);
400ff7:       \textcolor<2>{red}{bf 30 00 00 00}          mov    \$0x30,\%edi
400ffc:       e8 1f 05 00 00          callq  401520 <op\_create>
401001:       e9 20 ff ff ff          jmpq   400f26 <ts\_next\_token+0x46>
401006:       66 2e 0f 1f 84 00 00    nopw   \%cs:0x0(\%rax,\%rax,1)
40100d:       00 00 00
 125: p\_state->last\_token\_is\_od = 0;
401010:       c7 43 \textcolor<2>{red}{08 00 00} 00 00    movl   \$0x0,0x8(\%rbx)
 126: return (void*)op\_create(op);
401017:       \textcolor<2>{red}{bf 04 00 00 00}          mov    \$0x4,\%edi
40101c:       e8 ff 04 00 00          callq  401520 <op\_create>
   \end{semiverbatim}
   \end{Example}
\end{frame}
  % file:line maps to an invalid address
  % ptrace: Does not set breakpoints, modifies memory
  % ptrace is not recursive, can't strace kcov
  % When a process stops, you need to dump the register file and find the PC. This is
  % architecture and OS dependent

\section{Other kcov features}

\begin{frame}{So how does Python and Bash work?}
\end{frame}

\subsection{System mode}
\begin{frame}{System mode}
\end{frame}

\end{document}

%Kcov - a single-step code coverage tool

%- Motivation

%  - What is code coverage?

%  - goto fail

% - Outline of the presentation


% - Setting: UNIX


% - Problems with gcov + lcov etc

%  - Multiple steps

%  - Loses data on program crash


%- Basic demo of kcov

%  - Explain how the output looks, 1/2 etc

%- History

%  - Bcov by Thomas Neumann

%  - Kernel kcov - not related

%- Some features

%  - What I use if for

%  - Demo with filtering, merging etc


%- How kcov works

%  - A bit about how gcov works

%  - Breakpoints

%  - Differences between OSX and Linux/FreeBSD

 % - Complain a bit about elf_version/libdwarf etc.


% - What about speed?
