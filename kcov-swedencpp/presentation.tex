\documentclass{beamer}

\usepackage[utf8]{inputenc}
\usepackage{pgfpages}
\usepackage{listings}
\setbeameroption{show notes on second screen}

\usetheme{Warsaw}
\graphicspath{ {./images/}{../images/} }

\AtBeginSection[specialframe]
{
  \begin{frame}{Table of Contents}
   \tableofcontents[currentsection]
  \end{frame}
}

\title[Kcov - a single-step code coverage tool] %optional
{Kcov - a single-step code coverage tool}

%\subtitle{A short story}

\author{Simon Kågström}

\institute
{
  Net Insight\\
}

%\logo{\includegraphics[height=1.5cm]{lion-logo.png}}

\begin{document}

\frame{\titlepage}

\begin{frame}
  \frametitle{Motivation}
  \includegraphics<1>[height=8cm]{goto_fail_no_coverage}
  \includegraphics<2>[height=8cm]{goto_fail}
\end{frame}

\begin{frame}{Outline}
  \tableofcontents
  \note{Unix environment}
\end{frame}

\section{Overview}
\subsection{Problems with traditional tools}
\begin{frame}[fragile]{Problems with traditional tools}
  \begin{itemize}
  \item gcov + lcov is a multi-step process
  \item gcov leaves droppings after compilation/running
  \item A program which crashes will not generate coverage data
  \end{itemize}
  \begin{Example}
    \begin{semiverbatim}
     \scriptsize
\$ gcc -g -Wall --coverage goto-fail.c
\$ ./a.out
\$ ls
  a.out  goto-fail.c  goto-fail.gcda  goto-fail.gcno
\$ lcov --capture --directory project-dir --output-file coverage.info
\$ genhtml coverage.info --output-directory out
    \end{semiverbatim}
  \end{Example}
\end{frame}

\subsection{Kcov overview and demo}
\begin{frame}{Kcov overview}
  \begin{itemize}
    \item kcov
    \item Interactive Demo!
  \end{itemize}

  \note{
    \begin{itemize}
      \item Make sure /tmp/kcov is empty

      \item kcov /tmp/kcov projects/build/kcov/target/src/kcov

      \item Show result in browser

      \item Show how a single file looks, describe 1/3 etc

      \item Show file list again, note /usr/include etc

      \item Remove /tmp/kcov
  \end{itemize}
  }
\end{frame}

\subsection{Main features of kcov}
\begin{frame}{Features}

  \begin{itemize}
    \item kalle
  \end{itemize}

  \note{
    \begin{itemize}
      \item Run kcov with --include-pattern, note that --exclude-pattern and paths also exist

      \item Run kcov on another program (calc?), show that output is merged
    \end{itemize}
  }
\end{frame}

\section{How kcov works}
\subsection{Overview}
\begin{frame}{How kcov works}
\end{frame}

\subsection{Elves, dwarves and breakpoints}
\begin{frame}{Dwarf stabs Elf}
  \begin{itemize}
    \item \textbf{elf\_version} must be called at program start!
  \end{itemize}
\end{frame}

\subsection{Implementation quirks on different architectures}
\begin{frame}{kcov on Linux, FreeBSD and Mac OSX}
\end{frame}

\subsection{ELF/Dwarf implementation quirks}
\begin{frame}{Libelf}
\end{frame}

\section{Other kcov features}

\subsection{Integration with CI systems}
\begin{frame}{Integration}
  \begin{itemize}
  \item<1-> Kcov can produce output for multiple CI systems
  \item<1-> Uploading to Coveralls.io is built-in
  \item<2-> Uploading to Codecov.io is supported by the upstream site
  \end{itemize}
  \includegraphics<1>[height=6cm]{coveralls}
  \includegraphics<2>[height=6cm]{codecov}
\end{frame}

\subsection{Python/Bash}
\begin{frame}{Python}
\end{frame}

\subsection{System mode}
\begin{frame}{System mode}
\end{frame}

\end{document}

%Kcov - a single-step code coverage tool

%- Motivation

%  - What is code coverage?

%  - goto fail

% - Outline of the presentation


% - Setting: UNIX


% - Problems with gcov + lcov etc

%  - Multiple steps

%  - Loses data on program crash


%- Basic demo of kcov

%  - Explain how the output looks, 1/2 etc

%- History

%  - Bcov by Thomas Neumann

%  - Kernel kcov - not related

%- Some features

%  - What I use if for

%  - Demo with filtering, merging etc


%- How kcov works

%  - A bit about how gcov works

%  - Breakpoints

%  - Differences between OSX and Linux/FreeBSD

 % - Complain a bit about elf_version/libdwarf etc.


% - What about speed?
